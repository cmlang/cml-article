\section{Related Work}\label{sec:related}

The following subsections compare CML to other languages, tools and frameworks
that can also generate code from conceptual models.
Each subsection covers a different category,
enumerating specific solutions and characterizing their relevance to CML, and also their differences. 

\subsection{Textual Languages}

This subsection covers existing text-based languages designed to enable code generation from conceptual models.
When compared to CML, the following languages are the most relevant,
because they are all textual languages designed for code generation:

\begin{itemize}
\item MPS
\item Xtext/Xtend
\item M Language
\item MM-DSL
\item XML/XSLT
\item IFML
\end{itemize}

Observe that most of the solutions listed above enable modeling via DSLs, while CML is a generic language for modeling in any domain.

\subsection{Graphical Languages}

This subsection covers existing graphical languages designed to enable code generation from conceptual models.
Despite CML being a textual language,
the following graphical languages still have some relevance,
because they have also been used to generate code in other target languages:

\begin{itemize}
\item MPS
\item FCML
\item MetaEdit+
\item MDA
\end{itemize}

The major drawback behind graphical languages,
as covered in section \ref{sec:why},
is their inability to integrate seamlessly with the workflow, tools and mindset of software developers.

\subsection{Frameworks}

Frameworks allow code generation from conceptual models, but lack a modeling language -- graphical or textual. 
EMF is a classical example,
where modeling is done via editors on Eclipse or via a programming interface,
and the models are stored in the ECORE/XML format. 

Frameworks may also be used as the infrastructure of modeling languages.
EMF, for example, is the framework supporting Xtext.
Conceivably, other modeling languages may also target EMF.
In fact, CML's extensible compiler allows the implementation of templates that target EMF.
