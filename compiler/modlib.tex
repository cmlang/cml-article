\subsection{Modules and Libraries}\label{subsec:modlib}

When developing a single application with just a few targets,
having a single directory to maintain all the CML source code is sufficient.
But, once more than one application is developed as part of a larger project,
and CML model elements are shared among them,
it is necessary to separate the common source code.
Also, some applications cover different domains,
and it may be beneficial to separate the source code into different CML models.

In order to allow that, CML supports \emph{modules}.
Grouping a set of CML model elements,
a module in CML is conceptually similar to a UML \cite{uml} package.
Physically, each module is a directory containing three sub-directories:

\begin{itemize}
\item \emph{source}: where the CML source files reside.
\item \emph{templates}: optional directory containing templates for code generation.
\item \emph{targets}: created by the CML compiler to contain each target sub-directory, which in turn contains the target files generated for a given target.
\end{itemize}

Under the \emph{source} directory, the module is defined by a \emph{module specification}.
If a module needs to reference CML model elements in other modules,
then an import statement defines the name of the other modules.
The CML compiler will then compile the imported modules
before compiling the current module.
In order for the CML compiler to find the other modules,
they must be in a sub-directory with the module's name
in the same directory where the current module is placed.

CML modules have no versions
as they are maintained in the same code repository with the other modules they import.
However, one can package a module as a library,
which will have a version and the same name as the module.
This library in turn can be published into a public (or company-wide) \emph{library site}
in order to be shared with other developers.
A CML library is then a packaged, read-only module with semantic versioning \cite{semver},
where compatible versions do not change or remove public elements of the library's CML model (or function/parameters from the library's templates); only add new elements.
Fixes cannot change the library's public elements; only internal elements.
These rules may be enforced by the CML compiler when packaging and publishing new versions of a library, and they are essential to guarantee the evolution and reuse of CML models.
