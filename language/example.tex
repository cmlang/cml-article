\subsection{An Example}\label{subsec:example}

On the example of figure \ref{fig:store}, some concepts, such as \emph{Book} and \emph{Customer}, are declared in CML.
The block-based syntax declaring each concept resembles the C \cite{clang} language's syntax.
Each concept declares a list of properties.
The property declarations are based on the Pascal \cite{pascal} style for variable declarations,
where the name is followed by a colon (``:'') and then the type declaration.
Part of the CML syntax for expressions, such as the expression in \emph{BookStore}'s \emph{orderedBooks}, is based on OCL \cite{ocl} expressions.
While the syntax of the expression in \emph{goldCustomers} is new,
its semantics also match OCL \cite{ocl} query expressions.

\begin{figure}
\verbatimfont{\scriptsize}
\begin{verbatim}
concept BookStore
{
    books: Book+; customers: Customer*; orders: Order*;
    /goldCustomers = customers | select totalSales > 1000;
    /orderedBooks = orders.items.book;
}

concept Book
{
    title: String; price: Decimal; quantity: Integer = 0;
}

concept Customer
{
    orders: Order*;
    /totalSales = orders.total | sum;
}

concept Order
{
    customer: Customer;
    total: Decimal;
}

association CustomerOrder
{
    Order.customer: Customer;
    Customer.orders: Order*;
}
\end{verbatim}
\caption{Adapted from the fictional Livir bookstore; a case study by Wazlawick \cite{wazlawick}.}
\label{fig:store}
\end{figure}


The key language features are:
\emph{Book} and \emph{Customer} are concepts;
\emph{title} and \emph{price} under the \emph{Book} concept are attributes;
\emph{totalSales} under the \emph{Customer} concept is a derived attribute;
the properties \emph{books} and \emph{customers}
declared under the \emph{BookStore} concept
represent unidirectional associations
(in UML \cite{uml}, they would correspond to the association roles);
\emph{CustomerOrder} binds two unidirectional associations
(represented by the \emph{orders} property under the \emph{Customer} concept
and by the \emph{customer} property under the \emph{Order} concept)
into a single bidirectional association;
the properties \emph{goldCustomers} and \emph{orderedBooks}
under the \emph{BookStore} concept are examples of derived associations.

These language features are defined in the subsection \ref{subsec:metamodel}.
