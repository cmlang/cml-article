\section{Why A New Language?}\label{sec:why}

At this point, one could question: why define a new textual language for conceptual modeling from scratch? Instead of basing it on an standard, richer textual language, such as OWL 2 \cite{owl2}. There are two primary reasons explained in the follow subsections:

\subsection{Developer Experience}

CML is intended to enable software developers to do \emph{conceptual modeling} in the same workflow they are used to doing \emph{programming}.

The Manchester \cite{owl2manchester} syntax of OWL 2 is intended to be user-friendly,
but it still does not resemble the syntax of commonly used, block-based, imperative programming languages,
such as C \cite{clang} and its syntax-alike descendants.
Manchester's syntax is also unlike the syntax of declarative, query languages, such as SQL [?].

Using CML,
and its familiar syntax (as we shall demonstrate in the next sections),
it is expected that developers will be able to raise the abstraction level of their programs,
and still work with a syntax that is familiar to them.
By providing its own syntax,
CML then promotes the \emph{modeling-as-programming} approach being presented by this work. 

\subsection{Language Evolution}

CML is expected to evolve with its compiler, and the tooling around it.
Unlike the expressive power seen on OWL 2 \cite{owl2} with its breadth of features,
and its ambition to model complete ontologies for the Web,
the CML language and its extensible compiler intentionally support a limited number of features and scenarios.

This first version has been designed for the initial validation of the model-driven development approach taken by CML.
As developers provide feedback,
new language features may be added in order to enable the extensible CML compiler to support new modeling/development scenarios.

