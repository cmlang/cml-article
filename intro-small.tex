\section{Introduction}
%
In order to address the challenges of the ever-changing,
increasingly distributed technologies used on software systems,
Model-Driven Architecture (MDA \cite{mda}) from Object Management Group (OMG)
has been promoting model-driven software development.
In particular, MDA has guided the use of high-level models
(created with OMG standards, such as UML \cite{uml}, OCL \cite{ocl} and MOF \cite{mof})
to derive software artifacts and implementations via automated transformations.
As one of its value propositions, the MDA guide \cite{mda} advocates:

\begin{quote}``Automation reduces the time and cost of realizing a design,
reduces the time and cost for changes and maintenance and produces results that ensure consistency across all of the derived artifacts.''\end{quote}

MDA provides guidance and standards in order to realize this vision,
but it leaves to software vendors the task of providing the tools
that automate the process of generating the implementations from the models.
The key role played by tools has been demonstrated by Voelter \cite{voelter}
in his \emph{Generic Tools, Specific Languages} approach for model-driven software development.
Voelter \cite{voelter} has used domain-specific languages (DSLs) with the Metaprogramming System (MPS)
in order to generate software artifacts.
Unlike MDA, which is based on UML/MOF models,
MPS allows the specification of models using domain-specific editors.

The conceptual modeling language and extensible compiler presented here are an alternative approach to MPS.
While the latter is a fully integrated development environment
based on domain-specific languages and their projectional editors\footnote{Projectional
editors in MPS do not rely on parsers.
Instead, the abstract syntax tree (AST) is modified directly.
MPS renders the visual representation of the AST based on the DSL editor definition.},
the former (hereby called CML) is a compiler.
CML has, as \emph{input}, source files defined using its own conceptual language,
which provides an abstract syntax similar to (but less comprehensive than)
a combination of UML \cite{uml} and OCL \cite{ocl};
and, as \emph{output}, any target languages,
based on extensible templates that may be provided by the compiler's base libraries,
by third-party libraries, or even by developers.
As part of the author's Computer Sciences Bachelor Technical Report,
both the CML language and compiler are in its initial stage of development,
and available as an open source project online \cite{cml-repo}.

Section \ref{sec:why} explains the motivation for creating yet another language for conceptual modeling.
The next two sections present the language (section \ref{sec:lang})
and the compiler with its extensible templates (section \ref{sec:compiler}).
Section \ref{sec:related} compares CML to other languages, tools and frameworks
that can also generate code from conceptual models.
We conclude in section \ref{sec:conclusion},
reiterating the objectives being pursued by CML
and exploring options to validate the use of the CML compiler.
